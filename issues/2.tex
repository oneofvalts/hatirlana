\documentclass[a4paper, twocolumn]{memoir}

\setulmarginsandblock{2cm}{2cm}{*}
\setlrmarginsandblock{2cm}{2cm}{*}
\checkandfixthelayout{}

\usepackage[pdftex, pdfpagelabels, bookmarks, hyperindex, hyperfigures]{hyperref}

\usepackage{tgpagella}
\usepackage[T1]{fontenc}

\usepackage[turkish]{babel}

\author{
  uyu, abdullah\\
  \texttt{oneofvalts@sdf.org}
  \and
  uyu, bayram\\
  \texttt{m.bayramuyu@gmail.com}
  \and
  uyu, durmus\\
  \texttt{durmusuyu@gmail.com}
}
\title{\textsc{hatirlana}\\\small sayı: 2}
\date{28 şubat 2021}

\nouppercaseheads{}
\setsecnumdepth{chapter}

\settowidth{\versewidth}{param bahçe}

\begin{document}
\maketitle
\section{l'authenticité}
doğan cüceloloğlu adını eskiden beri duysam da veyis ateş’in hatırda
kalan programında esasen tanımış sevmiştim. bizim yazı dizimize de uydu.
şivesinden utanmayan kendi kelimelerini kullanan bundan gocunmayan bir
adam, yörük. geldiği yerden utanmıyor. dışındaki kabukları attığında
gocabbama çok benzettim. saflığı, iyi niyetliliği, sevmesi, üzülmesi.
bizi biz yapan şeylerde çok ortak noktamız var bunu biliyorum. bu ay
hakkın rahmetine yürüdü. yattığı yer nur olsun.
\section{les standards de cogna}
mustafa sami onay. sonunda gidip tanıştım. konya kaşığının son ustası.
konya kaşığı diyorum çünkü anadolu kaşıkçılığı’nın son ustası diyerek
kıymetini azaltmak istemedim. konya’nın aklınıza gelmeyecek işlerde çok
iyi işler çıkardığını görmek şaşırtmaya devam ediyor. hatta bir işi iyi
yapmanın adına artık konya standartlarında diyeceğiz :) neyse mustafa
abiden biraz kaşık aldım örnek. yaptığım kaşıkları da gösterdim. beğendi,
bu da beni çok mutlu etti. bir kaşığını taklit etmeye çalıştım. kaşığın
anatomisi o kaşığı taklit ederken çok iyi anlaşılıyor. ben kaşıklara aşık
oldum. aydın abinin; bayram bu kaşıklar avrupa kaşıkları, sen bize
konyalıların kuşaklarında taşıdıkları kaşıktan yap dediği kaşıktan aldım
iki tane. inşallah mustafa abiyle hukukumuz yıllar geçtikçe artar.
\section{déjà}
emrak ablak ve oturmak. diyecektim de abdullah yazmıştı bunu :)
\section{l'échec}
satranç. bu oyun denilince aklıma ilk muhammed han geliyor ve hiç
aklımdan çıkmayan açılışı. şimdi olsa oyununa gelmem tabi. acaba hala
oynuyor mu? ikinci defa stefan zweig’in satranç kitabıyla gündemime
girdi. ve sonunda abdullah’ın israrları sonuç verdi ve satranç artık
hayatımın önemli bir parçası. bu akşam da turnuvamız var. yalnız baya
kötü oynuyormuşum onu farkettim ve lichess’deki puanların insanların
seviyelerini nasıl doğru yansıttığını. oyun seni seviyene çok hızlı
sabitliyor.
\section{naval}
askerlik. neredeyse askere gidiyordum bu ay. aslında tecil süremin
dolduğunu ve kaçak duruma düşeceğimi ve dahî para cezası olduğunu bilmeme
rağmen e-devlet’ten gelen bir mesajla irkildim. her şeyi ölçtüm tarttım,
yok olmuyor gidemiyorum. hatay iskenderun deniz kuvvetleri komutanlığı.
bahriyeli er bayram uyu. kulağa da hoş geliyor hani :)
\end{document}
