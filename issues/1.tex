\documentclass[9pt, a5paper, twocolumn, openright]{memoir}

\setulmarginsandblock{4em}{4em}{*}
\setlrmarginsandblock{2em}{2em}{*}
\checkandfixthelayout{}

\usepackage[pdftex,
pdfpagelabels,
bookmarks,
hyperindex,
hyperfigures,
colorlinks]{hyperref}

\usepackage[T1]{fontenc}
\usepackage[sfdefault]{AlegreyaSans}
\renewcommand*\oldstylenums[1]{{\AlegreyaSansOsF #1}}

\usepackage{inconsolata}

\usepackage[turkish]{babel}

\nouppercaseheads{}
\setsecnumdepth{chapter}

\settowidth{\versewidth}{param bahçe}

\begin{document}
\thispagestyle{plain}
\noindent
{\HUGE\textsc{hatirlana}\\\small{}sayı: 1\\ocak 2021}\\\\
{\small{}abdullah uyu\\\texttt{oneofvalts@sdf.org}}
\bigskip
\section{masa}
emrah ablak herhangi bir fikrin doğumuna masasının başında vakit
geçirmeği şart koşuyor. onun için, masadan ayrılmak düşün dünyayı
terketmek demek. hiçbir şey yapmıyor olmak dahî sorun değil çünkü ancak
masada tefekkür edilebilirdi zaten. saatleri oturarak tüketdikden sonra
birşeylerin mecburen çıkageleceğine emin.

bu sebeple tersine bir usavurumla, masa başında geçen vakdin boşu da yok.
çalışmak sürecinin tanımında düşünmek çok büyük cüz.
\poemtitle{kıvran dalda --- bir taze koşma}
\begin{verse}[\versewidth]
  \itshape{}
  kıvran dalda\\
  hava soluk\\
  fikir sağda\\
  seni sevdim
\end{verse}
\section{öğrenmek}
gilbert strang'e ders işleyişinde dikkat etdiği şeylerden sorulduğunda
öğrenciyle aynı düşünce zinciri üzerinde olmakdan, beyinlerin birbirine
teğet olması gerekliliğinden bahsediyor. birşey açıklarken karşıdakinin
düşün sürecin neresinde olduğunu umursamamak büyük bir saygısızlık ve
eğitimbilimsel hata.
\begin{verse}[\versewidth]
  \itshape{}
  bir kere bil\\
  iki kere\\
  susmak gerek\\
  kere bazı
\end{verse}
strang bu minvalde, derslerini anlatırken daha önce anlamış
olduğunu seslendirmek yerine ona mukabele edip başdan başlıyor ki o da
çevresindeki herkes gibi bir öğrenciye dönüşsün. böylelikle ders,
herkesin yolcu fakat birinin gidilicek yolu bilen olduğu bir yolculuk.
yanlış olan yolun sonundan yol arkadaşlara bağırmak, doğru olan onlarla
beraber yolu yürümek.
\begin{verse}[\versewidth]
  \itshape{}
  kalem sazı\\
  sev çok tezi\\
  kaşı keman\\
  güler bazı
\end{verse}
\section{kalem sazı}
yazmak kalem denen sazı çalmak, mürekkep de mızrap ola. kağıt hava olmalı
ve dolayısıyla yazı nağme. kalem ucu tel, damağı gövde. ucu ayarlamak da
akort etmek olsa gerek. sıradan kağıtlar kullanamayışımız da böylece
anlaşılır: tambur mesela, bir oda sazıdır, açık havada çalmakla aynı
deneyim elde edilmez, bir kalem de sıradan bir kağıt değil de tomoe river
ister. hadi claire fontaine veya rhodia olsun.
\begin{verse}[\versewidth]
  \itshape{}
  al mürekkep\\
  apapak ten\\
  derler yazmak\\
  çal o sazı
\end{verse}
elimizde bir twsbi eco, bir pilot kaküno ve modelini bilmediğimiz bir
parker kalem var. akort etmek işini tablet ilaçları korumak için
kullanılan ince metal yapraklar ile yapmak mümkün. denediğimiz ve
başarılı olduğumuz için söylüyorum. sırf bu iş için ince bakır yapraklar
satılıyor da.
\poemtitle{kölge\~n}
\begin{verse}[\versewidth]
  \itshape{}
  dolan dur o makamdan\\
  ne in ne çık andan\\
  sırsa da gizledi\~n\\
  dağılsam da parlak uzak\\
  kölge\~ne bakanda
\end{verse}
\end{document}
