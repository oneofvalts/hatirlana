\documentclass[9pt, a5paper, twocolumn, openright]{memoir}

\setulmarginsandblock{4em}{4em}{*}
\setlrmarginsandblock{2em}{2em}{*}
\checkandfixthelayout{}

\usepackage[pdftex,
pdfpagelabels,
bookmarks,
hyperindex,
hyperfigures,
colorlinks]{hyperref}

\usepackage[T1]{fontenc}
\usepackage[sfdefault]{AlegreyaSans}
\renewcommand*\oldstylenums[1]{{\AlegreyaSansOsF #1}}

\usepackage{inconsolata}

\usepackage[turkish]{babel}

\usepackage{dirtytalk}

\usepackage{float}
\usepackage{needspace}

\nouppercaseheads{}
\setsecnumdepth{chapter}

\addto\captionsturkish{\renewcommand{\figurename}{şekil}}
\counterwithout{figure}{section}

\settowidth{\versewidth}{param bahçe uydular ordularla geçti günler}

\begin{document}
\thispagestyle{plain}
\noindent
{\HUGE\textsc{hatirlana}\\\small{}sayı: 4\\nisan 2021}\\\\
{\small{}abdullah uyu\\\texttt{oneofvalts@sdf.org}}
\bigskip
\poemtitle{çağırdılar --- bir taze yolculuk}
\needspace{9\baselineskip}
\begin{verse}[\versewidth]
  \itshape{}
  çağırdılar gittim, çağırdılar gittim\\
  bazı sağdan bazı soldan.\\
  bazen elden bazen koldan ağrılarla\\
  sanrıları izledik.\\
  ``bu bağırmalar, bu bağırmalar. nereye?'' sordum.\\
  anlaşılan küçük aklım boşa yordum.\\
  bu çok aşikar da, şaşırıyor muyum,\\
  yoksa bahşedildi mi bana bu saklı?\\
  yoksa bunları hiç tanımıyor muyum?\\
\end{verse}
\section{super dimensional fortress: a bbs}
sdf shell'inde {\ttfamily `bbs'} komutu var ve çalışdırıldığında size şu
çıkdıyı veriyor: {\ttfamily `BBS  Documentary   DVD  set  (limited signed
  and  numbered edition  for SDF users!) [...]'}. bu ipucuyla youtube'yi
taradım, anlaşılan birisi bu çok parçalı uzun belgesi birleşdirmiş ve
yüklemiş, aşağı yukarı beş saat. bulduğum gece bayağı heyecanlandım,
yaklaşık bir saatini izledim, uykum geldi yatdım. fakat gerçekden,
izlediğim kısım boyunca, nostalji ve merakın eksildiği bir an olmadı,
neredeyse hep heyecanlı kaldım. sürekli durdurup anlatılanları tekrar
dinlediğimi hatırlıyorum. fakat bu belgeseli, 80'lerde bir bbs'ye telefon
çevirerek bağlanmış birisi gibi izlemem de imkansız tabii. ben hiç
yaşamadığım bir çağ ve dünyanın nostaljisini yaşıyorum bir bakıma.
hatta, bu nasıl oluyor, ona da bir yandan şaşırmıyor değilim. sonuçda
ben, bugün tekrar bbs kullanmaya başlamış bir sdf kullanıcısı değilim,
gerçekden sonradan bir eklentiyim ve, adeta bir kitap okur gibi, bir film
izler gibi sdf ve bbs'yi yaşıyorum.
\needspace{6\baselineskip}
\begin{verse}[\versewidth]
  \itshape{}
  önceleri bana hoş kokular gösderdiler.\\
  sonra hep birlikde pest sesler dinledik.\\
  andan biçare bizler, acımasız nağmelere inledik.\\
  biraz olsun sakinleşip durunca,\\
  ``yol.'' dediler, devam etdik,\\
  sanrıları izledik.\\
\end{verse}
ilginç olan, yaşayabiliyorum: bugün
siz de, \texttt{telnet} programı mevcut herhangi bir bilgisayarın
uçbiriminden \texttt{`telnet sdf.org'} ile sdf'ye bağlanıp kendinize bir
hesap oluşdurabilirsiniz.

ben sdf'yi, arama motoruna \texttt{free unix shell} yazarak buldum.
amacım bilgisayarım yanımda değilken, etrafımda unix-benzeri bir sistem
yokken, uzakdan, temel hesap işlerimi gerçekleşdirmekdi. böyle bir
hizmetden beklentilerim şunlar:
\begin{itemize}
  \item bir metin düzenleyici, \texttt{vi} olmayan bir unix-benzeri
  sistem olmadığını düşünürsek doğrudan elde edeceğim, varsa da zaten
  bana uzak olsun.
  \item bir dosya sistemi, mail, not, veya çalışdırılabilir dosya tutmak
  için
  \item mümkünse bir mail servisi, mail okuyucu ve besteleyici
  \item belki bir \texttt{c} derleyici.
\end{itemize}
ana fikrim anlaşılmışdır zannediyorum, unix-benzeri olması, özünde,
yeterli ve şart koşul. fakat bir yandan, yanlış hatırlamıyorsam, doğrudan
bir bbs aradığım bir dönem de oldu ve bu yukarıda anlatdığım dönemle de
bir kısmı kesişiyor olmalı galiba. işde asıl o merakımın nereden
doğduğuyla ilgilenmekle bambaşka bir yere varacağız: halt and catch
fire, cameron howe, mutiny. ilk neyden sebep keşfetdim bir türlü
hatırlamıyorum, ama bir yerden sonra hepsi birlendi, yapboz tam oldu.
\needspace{7\baselineskip}
\begin{verse}[\versewidth]
  \itshape{}
  bir ara renklerden bahsetdiler,\\
  ``bunlar,'' dediler, ``yaramaz şeylerdir.''\\
  ``ancak ses olmasa, bi düşün, n'eylerdin?''\\
  senden bahsedemedim onlara,\\
  sanki yeri değil gibiydi. bu gösterdikleri de\\
  bir bakıma, hislerimin dibiydi.\\
  bildiler hep sevgimi sana deyip, susdum.\\
\end{verse}
\subsection{anonradio}
sdf'nin icecast radyosu, muazzam bir yer, sıradışı programlar ve en
önemlisi sıradışı bir tavır ve tarz. sebep: topluluk radyosu olması. sdf
kullanıcıları radyo yayını yapmak istedikleri için kurulmuş olsa gerek
anonradio, diğer tüm sdf projeleri gibi. haftalık programda bir sürü boş
slot var, düzenli konuşacak bir şeyiniz varsa programa dönüşdürebilir,
anonradio'da yayın yapabilirsiniz. daha düzensiz, daha az özenle yapmak
istediğiniz yayınlar \texttt{openmic}'de yapılabilir. günün belirli
vakitleri openmic için ayrılıdır. biri dizini koşmamızşa dosya
sisteminizden bir dizine veya bilgisayarınızın bir ses cihazına koşup
yayın yapabilirsiniz, siz yayın yaparken dizin kilitlenir, bir başkası
koşamaz. tamamen açık mikrofon: hakikaten herhangi biri bağlanabilir,
yukarıdaki prosedüre riayet etdiği takdirde.
\needspace{8\baselineskip}
\begin{verse}[\versewidth]
  \itshape{}
  ben böyle düşüncelere dalmışken\\
  çok yol gitmiş olmalıyız, her yer değişmiş,\\
  tek eski bildiklerim kalmış,\\
  bir an yabancı gibiydim.\\
  her biri birer laf etdi gene,\\
  yeniyi tanıtdılar. atılan her adımda adeta,\\
  pür-i pak kanıtdılar. ben müşahede etdim,\\
  sanrıları izledik.\\
\end{verse}
teknik talimatlar için \url{http://anonradio.net/openmic/} ve
\url{http://sdf.org/?tutorials/anonradio-dj} adreslerini ziyaret ediniz.
\section{gopher}
\texttt{attaque-cinq.com} artık port \texttt{70}'den bir gopherhole
sunuyor. ismini \texttt{`a needle \& a thread'} koydum.
\texttt{usavurdu} isimli derlemedeki tüm şiirlerimi burada ayrı ayrı
metin dosyası olarak sunuyorum. \texttt{lynx} ile \texttt{`lynx
  -assume-charset=UTF-8 gopher://attaque-cinq.com'} komutunu
çalışdırarak benim gopherholemi düzgünce görüntüleyebilirsiniz.
diğer türlü türkçe karakterler yanlış gözükecek, diğerlerinin
görüntüsünü bozacak.

hatırlana'ları gopherhole'me yüklüyeceğim, kaynak kodlar ve
yazdırılabilir dökümanları. ayrıca yepye\~nile bir \TeX/\LaTeX{} öğrenmek
macerasına atılacağım, bu macerayı da orada bir tematik jurnal halinde
belgeleyeceğim. iyi tanımlı olması için spesifik, gerçekleşdirilebilir
amaçlar yazmaya koyuldum. şimdilik şunları not almak kafi:
\begin{itemize}
  \item Ti$k$Z\&PGF kitabının \say{A Picture of Karl's Students} bölümünü
  okumak
  \item instagram'da \texttt{@}'a kadar paylaşdığım fotoğraflar için bir
  fotoğraf albümü tasarlamak, dizgilemek
  \item unix-benzeri sistemler, win ve macos'a \TeX{} dağıtımı kurmayı
  öğrenmek
  \item \LaTeX'in temellerini öğretmeyi öğrenmek, bunu yaparken özgün bir
  eğitim kitapçığı yazmak
  \item unix-benzeri sistemleri kullanmaya giriş kitapçığı yazmak,
  dizgilemek ve basıp izzet hoşgör, izzet hoşgör ve hasan hoşgör'e
  hediye etmek.
  \item \texttt{asus dsl-ac51} ve \texttt{zyxel vmg3312 b-10a v2} modem
  router'ler için dökümantasyon yazmak, dizmek.
\end{itemize}
\section{the office ve tamburla tanışmak geride kaldı}
gözyaşlarına gark olduk, bastık kahkahayı fakat, the office bitdi.
tamburda mahur peşrevin ilk hanesi ve teslimini kırık dökük de olsa
çalıyorum. suzidil saz semaisinin de ilk kısmını çıkardım galiba. aka
gündüz kutbay'ın nihavend taksimini bir yere kadar taklit edebiliyorum.
\say{biraz kül biraz duman (nihavend şarkı)}'ı çıkardım, biraz çalışmayla
çalıp söyleyebilirim zannediyorum.

geçen tamburu da alıp okula gitdim gezmeye. susumu tanabe ile denk
geldik, topoloji dersini dinledim, M. Kuga'nın Galois' Dream kitabını
takip ediyormuş. mola verdik, kahve içmeye indik, modal müzikden ve türk
müziğinde makamlardan bahsetdik biraz. \textit{démonstration} istedi, ona
nihavend ve saba, kısa vokal doğaçlamalar yapdım, ilgisini çekdi.
\textit{d harmonic minor}'u not aldığını hatırlıyorum, ha bir de segah'a
çok takıldı. belki de birkaç segah eser seçip bir derlemeyle gitmeliyim
tekrar yanına varırsam. kendisi temelde batı müziğiyle ilgilendiğini ve
keman, piyano çaldığını söyledi.
\section{kaşık hediye aldık}
durmuş uyu'yla, bayram uyu tarafından. yemek yapma kaşığı, şimşir.
bunun üzerine bayram uyu, daha, çok kaşık yapdı ve bu sonuncularda artık,
boğazı yapmayı ve inceliğini bir seviyeye kadar çözdü. son yapdığı
kaşıklar gerçekden çok güzel oldu.
\needspace{19\baselineskip}
\begin{verse}[\versewidth]
  \itshape{}
  ``burası yolun sonudur'' dediler, ``istersen''.\\
  istemem dedim, ``gidelim''.\\
  çûn bilmiyordum vardığımız yeri, anlatdım:\\
  ``belki bu, benim ayıbım,\\
  buradan geri dönecek yolu bilmiyorum,\\
  yol güzeldi fakat, şimdi ben kayıbım.''\\
  yine her birinin, gözünden birer yaş geldi.\\
  fakat benim gözüm yaşı, zannediyorum seldi,\\
  bana üzüldüler.\\
  ``tamam,'' dediler, ``yürüyelim çeşmeye''.\\
  daha varmadan yanına çeşmenin,\\
  uzağından gördüm oluğunu.\\
  düşdün aklıma gine, başladı derdin\\
  ıslak gönlüm deşmeye. hepimiz biraz içdik\\
  o çeşmenin suyundan, gözüme bakdılar.\\
  ``varalım,'' dediler, ``oturalım yanına'',\\
  aşikar etmedimse de onlara,\\
  demek sevgim ancak sana gizledim,\\
  sanrıları izledik.\\
\end{verse}
\begin{figure}[H]
\needspace{22\baselineskip}
\begin{verbatim}
a needle &          a'
    a thread by    aa
                  aa'
          aaa\    aa
        aa   aa  aa'
       aa    a'  'a'aaa\'  a/    a/
      aa     ||  aa     |  a'    a'
      aa     |   a'     |  a'    a'
      aa     /   aa    /   a'    a
        aaa_a     aa'aa     aaa_a'
                               /a'
                               aa'
                          a'  aa
                           \aaa

you have  reached aby's gopherhole.
you   will   find  literary   work,
thematic journals,  routes to other
people's places.

          email: oneofvalts@sdf.org
                     aby@tilde.club
       sdf sip/voip extension: 2154
\end{verbatim}
  \caption{attaque-cinq.com'daki gopherhole'nin root menüsü}
\end{figure}
\end{document}
