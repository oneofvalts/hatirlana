\documentclass[a4paper, twocolumn, openright]{memoir}

\setulmarginsandblock{2cm}{2cm}{*}
\setlrmarginsandblock{2cm}{2cm}{*}
\checkandfixthelayout{}

\usepackage[pdftex, pdfpagelabels, bookmarks, hyperindex, hyperfigures, colorlinks]{hyperref}

\usepackage{tgpagella}
\usepackage[T1]{fontenc}

\usepackage[turkish]{babel}

\usepackage{dirtytalk}

\nouppercaseheads{}
\setsecnumdepth{chapter}

\settowidth{\versewidth}{param bahçe uydular ordularla}

\begin{document}
\thispagestyle{plain}
\noindent
{\HUGE\textsc{hatirlana}\\\small{}sayı: 3\\28 mart 2021}\\\\
{abdullah uyu\\\texttt{oneofvalts@sdf.org}}
\bigskip
\poemtitle{ilk aşkın doğası}
\begin{verse}[\versewidth]
  \itshape{}
  ilk sevgi ilk şavk\\
  andan ateş-i aşkda geri köz kalır\\
  öçesiye sızılar ol ocak
\end{verse}
\section{eylençe}
\textit{sürüler içinde sürmeli goyun} hicaz, uğraşırke tespit etdi durmuş
uyu ile numan yüksel, musiki oturaklarımızın birinde. bu oturakların
sürekliliği için bir endişe duymuyorum artık. çünkü kovalamıyoruz bile.
dört sazımız var: bağlama, tambur, ney, elektrogitar. tambur aldım. ney
ve elektrogitar numan yüksel'in, bağlama durmuş uyu'nun. böylece
oturağımız, dört saz üç adamdan mürekkep. tek başına çalışmak, dinlemek
var, fakat beraber kazandığımız ivme her birimizi şaşırdıyor. taksimleri
taklit etmeye çalışıyoruz, bildiğimiz ne kadar ezgi varsa çalmaya
deniyoruz, eğlenceli, bağlayıcı. sazların akortlarıyla ilgileniyoruz,
herbirinden aynı notayı basmaya çalışıyoruz. ilginç gözlemlerimiz var,
sazları gün geçdikce daha iyi tanıyoruz.

bu oturaklarda bir huy gelişdirdik, oturakların ses ve video
kaydını alıyoruz. bunlar bir yandan hatıra belki, fakat birer video log da
aynı zamanda. bu serüven kaydetmeğe değer. kim bilir belki bir gün
\textsc{valts}'ın bir belgeseli çekilir ve orada kullanılır.
\begin{verse}[\versewidth]
  \itshape{}
  uyum hem kıyım\\
  derdi on dokuz düğüm bilmez sevgili\\
  kalmadı eylençem bugün benim
\end{verse}
\section{yürük semai formu}
bu forma bilinçli sevgim zekai dede'nin hicaz yürük semaisi
\textit{bülbül gibi pür oldu cihan nağmelerimden} ile başladı
zannediyorum. fakat ritmi keşfetdikden sonra geri dönüp, dinlediğim yürük
semailere bakınca münir nurettin selçuk'un nihavend yürük semaisi
\textit{ruhsarına aybetme nigah etdiğimi} var. andan bu bilinçle arayıp
bulduğum iki olağanüstü eseri not edeceğim:
\begin{itemize}
  \item dede efendinin ırak yürük semaisi \textit{hasretle temâm na'le
    döndüm sensiz}
  \item yine dede efendi'nin, acem-aşiran yürük semaisi \textit{ne
    hevâ-yı bâğ sâzed ne kenâr-ı kişt mârâ}
\end{itemize}
yürük semainin ağırlığından vazgeçip şarkı yazacaksak formdan bağımsız,
melodik bir yoğunluk ve biricikliği olmalı, ki o da gün sonunda bir yere
kadar tatmin ediyor beni. meylimdeki bu değişikliği kesin olarak ilk
farketdiğım eser de bence, zaharya'nın segah bestesi \textit{çeşm–i
meygunun ki bezm–i meyde canan dönderir}'dir. özellikle ibrahim reşit
çağın bey'in internetde de paylaşdığı 105 numaralı arşiv kaydından, münir
nurettin'in tüyler ürpertici icrası dikkate değer.
\newcommand{\textoverline}[1]{$\overline{\mbox{#1}}$}
\poemtitle{ben, benim, etrafım}
\begin{verse}[\versewidth]
  \itshape{}
  rahat yat\=am, temiz yat\=am\\
  ö\~nü benim, ardı benim\\
  adı\~n barnakları\~ndan gördüm dün\\
  sanım yok, seni\~n ün\\
  eğri benim, galgı benim
\end{verse}
\section{hafız kemal bey}
doğaçlamalara olan ilgimin pek yoğunca artmasının bir neticesi.
kendisinden dinleyip mest olduğumuz gazeller:
\begin{itemize}
  \item \textit{nigeh gülçîn-i hasret dâmenim pür-hâr-ı mihnetdir}, segah
  \item \textit{şimşir-i nigâhınla vuruldum ciğerimden}, uşşakda yegah
  \item \textit{feryâd ediyor bir gül için bülbül-i şeydâ}, nevâ
  \item \textit{nûr-i sînen, sîne-sûz-i şûle-i idrâk olur}, nihavend,
    müstakil değil, aynı makam şarkı \textit{bakmıyor çeşm-i siyâh
    feryâde} içerisinde
  \item \textit{okunur sihr-ü füsûn nûr-i nigâhında senin}, nihavend
\end{itemize}
yukarıda yazdığım yegah gazelin kaydında başka bir gazel veya herhangi
bir eser kaydında görmediğim müstesna bir şey oluyor. kemal bey,
icrasının sonunda \say{yaşa} diye nida atıyor, gerçekden çok yüksek ve
hacimli bir sesle. biraz sonra ilginç bir antonasyonla birisi \say{ne
bağırıyor yahu?} diyor ve bir başkası şaşkınlıkla \say{ne biliyim,
anlamadım.} diye cevap veriyor.
\poemtitle{nerden çatdım?}
\begin{verse}[\versewidth]
  \itshape{}
  sağda solda dolanırka\\
  iki gidip bi duruka\\
  nur-i nigahından daldım\\
  ben nası bi aşık oldum\\
  parlak kö\~nlü taşık oldum
\end{verse}
neva gazelin üzerine doğaçlandığı güfteyi daha önceden biliyordum, bir
hüseyni şarkı, subhi ziya bey'in. internetde \textbf{eskiplak} isimli bir
kanal tarafından paylaşılmış, münir nurettin'in, yine subhi ziya bey'in,
uşşak şarkısı \textit{gücendi biraz sözlerime münfail oldu} ile beraber
okuduğu bir kayıttan.

segah gazelin kaydının bahsetmeğe değer. bildiklerim arasından, kemal
bey'in sesinin en iyi duyulduğu kayıt galiba. doğaçlamanın kendi de
fazlasıyla rafine, kontrollü ve tam.

kemal bey, inanılmaz derecede net melodik sıçramalar yapabiliyor.
okuyuşda gözetdiğini zannetdiğim estetik kaygılar:
\begin{itemize}
  \item telaffuzda harflerin açıklık-kapalılığı. sesli harfler dikkati
    ilk çekenler, fakat sessiz harfler de düşün malzemesi
  \item nağmeler kelimelerden bağımsız değil, kelimenin tabiatına münasip
    nağmeler düşürülüyor. en azından, dejenere seviyede, melodik olarak
    alçalan mı yoksa yükselen mi nağme seçilecek, bunun kararı dahî ilk
    aşamada belki yeterli.
  \item yukarıdaki alt seviye kriterle kolayca nitelenemeyen durumlar
    için de genel bir kavramaya sahip olmakda fayda var. bununla
    beraber, ön tanımlı bir tavrın varlığı da önemli.
\end{itemize}
her ne kadar belli bir seviyeden sonra öznelliği ağır bassa da ilk husus
için, ilk başda düzeltilmesi değerlendirecek şeyler var, bir acemi için.
ikinci husus kendini yeterince açıklıyor. üçüncü için biraz şerh
yapılabilir. bahsedilen tavır için, mesela kemal bey'inkine vakarlı,
oturaklı sıfatlarını yakışdırıyorum. hafız sami bey'inki, diğer taraftan,
çok daha sert, kaba ve hiddetli. bu sıfatlar belki değişdirilebilir,
yahut belki sıfatlandırmakdan tamamen vazgeçebilirim de, bilmiyorum.
fakat kesin olan bir şey var: hem kemal bey'in hem de sami bey'in çok iyi
tanımlı tavırları var, özgünler ve kendilerini ifade etmek macerasında
çok yol katetmişler.
\begin{verse}[\versewidth]
  \itshape{}
  sabah olur başlar serim\\
  döndermeğe fikiri\~n\\
  akşam olan unudurum\\
  bir miyim ben iki mi\\
  bilmiyorum nerden çatdım ansızın\\
  ben nası bi aşık oldum\\
  parlak kö\~nlü taşık oldum
\end{verse}
\section{ye\~nile defter aldım}
önce üstündekileri doğrudan not edeyim:
\begin{quote}
\ttfamily
paper note,
notepad,
$ 210\times297 $ mm,
100 sheets,
60 grm ofset paper,
micro perforated,
squared,
note pad
\end{quote}
4 buçuk lira?! muhteşem alışveriş, bi de zaten emindim çok iyi kağıdı
olduğuna alırke, geldim eve bakdım, aynen de öyleymiş. yumuşak, iyi
seviyede yoğun ve yağlı. \texttt{feathering} pek az, bir seviyeye kadar
\texttt{ghosting} var fakat ben umursamıyorum, zaten kağıtlar bence tek
taraflı kullanılmalı. \texttt{bleeding} yok, veya pek nadiren var
diyebilirim. kalem pilot kaküno, mürekkep parker quink blue. \textbf{4
buçuk lira.}
\end{document}
